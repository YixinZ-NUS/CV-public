\documentclass{resume}
\ResumeName{张艺馨}

\usepackage{xcolor}
\usepackage{calc}

\definecolor{color1}{rgb}{0.22,0.45,0.70}  % light blue
\definecolor{color2}{rgb}{0.45,0.45,0.45}

\newcommand{\progressbar}[2][2cm]{%
    \textcolor{color1}{\rule{#1 * \real{#2} / 100}{1.5ex}}%
    \textcolor{color2!15}{\rule{#1 - #1 * \real{#2} / 100}{1.5ex}}}

\begin{document}

\ResumeContacts{
  \ResumeUrl{mailto:zyx83028016zyx@gmail.com}{zyx83028016zyx@gmail.com},%
  \ResumeUrl{https://github.com/YixinZhang2002}{github.com/YixinZhang2002},%
  \ResumeUrl{https://asoulfan.cc}{Personal Blog}%
}

\ResumeTitle

\section{个人总结}

\begin{itemize}
  \item 性别男,汉族。具备强大的自驱能力、工程实践能力。热爱技术复现与应用,同时注重人文体验。
  \item 热爱AI,有计算机视觉与嵌入式项目经验。熟练使用Linux进行运维开发。关注科技发展,追求探索有前景的未知事件,力求提高生产力。
参与APRU活动,培养国际视野与批判性思维。
  \item 认同开源文化,维护开源项目;联系布道者,参与腾源会、PapersWithCode、CodaLab、K8s、红帽读书会等社区。
\end{itemize}

\section{教育经历}

\ResumeItem
[哈尔滨工业大学|本科在读]
{哈尔滨工业大学}
[\textnormal{智能测控工程,电子与信息工程学院|} 本科在读]
[2020.09—2024.06(预计)]

\textbf{GPA: 3.2/4.0
%(学年前 15\%)
},无挂科。\textbf{2024年应届生}。

荣获2021、2022秋人民奖学金、2021-2022年度优秀学生称号。

英语应用流畅(六级成绩 610)(top 5\%),入选精英班(top30/ 800),校内课程均分95+。

“互联网+”大学生创新创业竞赛优秀奖,全国大学生学术英语词汇竞赛二等奖。

部分课程:Python 100、AI专家系统应用99 、ARM嵌入式92(top3)、微流控芯片实验室发展及应用90(top1)。
%

\section[技术能力]{技术能力}%\protect\footnote{蓝色进度条描述技能的熟练度。}
\begin{itemize}
\item[] Python,C++,计算机视觉(PyTorch,CNN/ViT),Git,DevOps,Office,Visio \hfill \progressbar{80} 
\item[] C,AI-NLP/LoRA/Adaptive、NeRF,Docker,MATLAB,Verilog,ARM开发,PCB,SPICE\hfill \progressbar{60} 
\item[] Java,Rust,Golang,RL/GNN/AutoML/Diffusion,嵌入式开发,RTOS,K8s,Ansible,前端 \hfill \progressbar{30} 
\item[] 正学习:测控总线技术、试验理论、GTC 2023、GitHub Galaxy 2023、GDC/ 计算机图形学、医学图像分析
\item[] 详见个人\ResumeUrl{https://asoulfan.cc}{技术博客}
\end{itemize}

\section{实习经历}

\ResumeItem{黑龙江省工业和信息化厅}

\begin{itemize}
  \item 暂且保密,欢迎后续联系
\end{itemize}

\section{科研项目}

\ResumeItem{视频模糊文字识别系统—基于\textbf{CNN}与\textbf{Transformer}}
[组员,架构设计]
[2021.09—2022.11] 

\begin{itemize}
  \item[] \textbf{项目获校一等奖,入选国家级大学生创新创业训练计划项目,结题得分92.6}(计算学部top3/ 50)。
  \item[] 开发端到端系统,包含目标检测-追踪-图像增强(超分/降噪等)架构。在部分数据集上获sota效果。

主要负责去模糊与文字识别部分。跟进业界发展,与导师商讨引入Transformer架构。创新地用单视觉模型同时解决特征提取和文本转录任务,最终提升PSNR指标1.3dB、英语识别准确率1.7\%,中文2.2\%。
  \item[] 严格执行项目管理甘特图,并协调每周举办组会,完成经典论文、CVPR与NeurIPS等顶会论文的分析学习。目前负责论文撰写与成果转化,拓展在医疗影像领域应用等。
\end{itemize}

\ResumeItem{\textbf{IETC:}基于深度学习的智能道闸控制系统}
[组员,算法部署]
[2022.04—2022.09] 

\begin{itemize}
  \item[] 项目代表校智能物联网技术俱乐部,参与2022年全国大学生物联网设计竞赛
(\textbf{华为杯})。
  \item[] 负责目标检测算法在ARM架构微处理器MM32F5270上的部署调优;结合先验信息对YOLOv5架构\textbf{模型压缩}。通过评估-选择-微调,显著减少计算量和参数量,优化提升实时运行帧数约30\%。
  \item[] 修改项目计划与项目进度管理甘特图模板,供俱乐部,实验室与项目组使用。
\end{itemize}

\ResumeItem{基于大数据的疫苗接种数据预测与预警}
[组长]
[2020.09—2021.08]

\begin{itemize}
  \item[] 原生项目,\textbf{获校二等奖}。执行项目立项、中期及结题答辩,与数据可视化展示。
  \item[] 组织学习Hadoop相关知识,并从多数据源获取疫情及疫苗数据。引入改进ARIMA模型阶段性预测COVID-19蔓延趋势,结合大数据平台对社会资源进行建模分析与协调管理,形成系统优化方案。系统短期预测准确率达90\%。

\end{itemize}

\ResumeItem{“三下乡”暑期社会实践}
[组长,策划]
[2022.06—2022.09] 

\begin{itemize}
  \item[] 率领团队探访大国重器。立项阶段入选哈尔滨工业大学社会实践\textbf{重点培育团队},结题获\textbf{校二等奖}。
  \item[] 依据防控形势充分准备预案。率领考察采访哈工大航天馆,总结组员后续独立探访成果,形成调研报告与新闻稿。

\end{itemize}

\ResumeItem{其他}
[参与“兴智杯”全国人工智能创新应用大赛(工信部),北京微电子技术研究所训练营(FPGA);\hspace*{2.6em} 短期编译移植TWRP Recovery,维护一机型社区;近期调试OpenHarmony富设备移植。\\经历皆可提供证明,欢迎邮件联系]


\end{document}
